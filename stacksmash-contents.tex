\title{Lab: Smashing the Stack For Fun and Learning}
\subtitle{A lab on arbitrary code execution in a locked down system}

\author{%
  Daniel Bosk
}
\institute{%
  Department of Information and Communication Systems,\\
  Mid Sweden University, SE-851\,70 Sundsvall
}

\maketitle


\section{Introduction}
\label{sec:intro}
Buffer overruns of different kinds are common vulnerabilities in software.
Examples vary from being able to break the DRM of a Nintendo Wii console 
\cite{twilighthack} to reading the memory of a running OpenSSL implementation 
\cite{heartbleed}.

How this really works is that the software mistakenly do not check the 
boundaries of buffers.
This might be due to incorrect assumptions by developers, the system being 
comlpex, or simply human error.

\subsection{Aim}
\label{sec:aim}
The aims of this assignment is to look into buffer overrun vulnerabilities.
The intended learning outcomes are as followin, after completion of this 
assignment you should be able to:
\begin{itemize}
  \item Understand the consequences of vulnerabilities in software.
\item Protect software from the easiest vulnerabilities.
\item Evaluate strengths and weaknesses in software design.

\end{itemize}

The next section covers what you must read before you understand this 
assignment and how to do the work.
\Cref{sec:tasks} covers the work to be done, i.e.~how you should learn this.
\Cref{sec:exam} covers how it will be examined, i.e.~how you show that you have 
fulfilled the intended learning outcomes given above.


\section{Theory}
\label{sec:reading}
To grasp this assignment you must first read
Chap.~4, 8, 9, 18 in \citetitle{Anderson2008sea}~\cite{Anderson2008sea} and 
then you must read Chap.~5--7 (and optionally 8), 10--12, 20, in 
\citetitle{Gollmann2011cs}~\cite{Gollmann2011cs}.

After reading the material given above you need to know some assembly 
programming, specifically x86-64 assembler and some tools.
For this you should read \citetitle{assembly} by \citet{assembly}.
You also need to be acquainted with some tools, for that reason, study the 
manual pages for objdump(1), as(1), and gdb(1).

Finally, you should read the main paper of this assignment: the classic paper 
on stack smashing, the first paper on the matter to be precise, 
\citetitle{stacksmash}~\cite{stacksmash}.



\section{Assignment}
\label{sec:tasks}
This assignment will use the scenario of a buffer overrun bug found in Sun's 
Solaris 8 and 9~\cite{passwdbug}.
The scenario is a vulnerability in the passwd(1) program which allows for 
arbitrary code execution.
We will use a much simplified version of the passwd(1) utility, the source code 
can be found in Sect.~\ref{app:passwdsrc} in the appendix.

The first part of the lab concerns exploiting a buffer overrun vulnerability in 
this program to execute a new shell.
This shell will, due to the nature of the passwd(1) utility, have root 
priviledges, since it can execute the setuid(2) system call to change the 
effective user ID to that of root.

This first part of the assignment will be solved together during a full-class 
hackathon in the computer lab.
There will be a projector with the code for all to see, then we will rotate who 
will be by the keyboard writing what the rest of the class is saying.
This way we will discuss together and write the code together, everyone will 
thus participate in the process.

The second part of the assignment is to dicsuss the consequences of this, among 
other things we will discuss the following questions:
\begin{itemize}
  \item How can we possibly detect if this attack has occured somewhere?
  \item How can we protect ourselves from vulnerabilities such as these?
\end{itemize}


\section{Examination}
\label{sec:exam}
To pass this assignment you must first actively participate in the hackathon 
lab session.
You must also actively contribute to the post-coding discussions.

If you cannot participate in the lab session you have to solve the lab 
yourself, then orally present your solution during one of the lab sessions 
after the course-end.


\subsubsection*{Acknowledgements}

This work is licensed under the Creative Commons Attribution-ShareAlike 3.0 
Unported license.
To view a copy of this license, visit 
\url{http://creativecommons.org/licenses/by-sa/3.0/}.
Its original source code can be found in URL 
\url{https://github.com/OpenSecEd/stacksmashlab/}.


\printbibliography{}


\appendix
\section{The simple passwd.c program}
\label{app:passwdsrc}
The following program will be used for experimenting with buffer overruns.
The source code is available in a file with the source code of this document, 
that source also contains a Makefile to build it properly (with disabled stack 
protection etc.).
See the repository on URL
\begin{center}
  \url{https://github.com/OpenSecEd/stacksmashlab/}.
\end{center}

\lstinputlisting[language=C,firstline=32]{passwd.c}

